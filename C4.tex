\chapter{算法工程师}

加油写啊

\section{本章小节}

接受的教育

找工作的经历

工作的终止

被裁员的经历

当老板以强提醒的方式ding我,并通知我去会议室时,我大概已经猜到发生了什么。毕竟,老板没事不会找我,入职以来只找我 one one 过两次——一次是刚入职沟通我的目标,二是年底沟通绩效。

推开会议室的门,桌上摆着几份纸质文件。原来之前公司内网传得沸沸扬扬的裁员新闻是真的!我 一下就明白了,“不是吧!”,虽然有点意料之内,但这种新闻落到自己头上,总归有点吃惊。我在老板身边坐下,他连说了几句“我实在没办法”。

现在回想那天的情形,仍旧很梦幻。本该十几分钟签完的文件,因为我的震惊和伤心,让老板又以过来人的身份对我多加安慰和教导,结果足足聊了一个小时。到底哪些是真话,哪些是官话,我已无从知晓。以下我将分析自己被裁的原因,以及分享一些在大厂学习到的工作经验。

为什么被裁员的是我,而不是别人?

1. 业务收缩,对公司而言性价比太低

当时拿着sp进来,每个月拿着税后2w+的工资,对于父母都是工薪阶层的我来说,简直就是实现了梦想。但是作为一个应届生,技能绝对不如一个工资被倒挂的老员工。如果公司还有钱可以烧,没事,可以培养,把应届生的潜力转化为实力。但是对于业务收缩的部门而言,这种性价比极低的员工已经不适合继续留在公司开拓业务了。

2. 没有寻找新的项目和突破点,成为了闲人

公司的业务是在不断变化的,多个业务线有交叉也有融合。做项目的时候如果不能做一个看两个,那么当自己的业务线被砍的时候,就成了一座孤岛。在我所在的部门经历了一波人事变动之后,项目也有所调整。但是我没有积极地找老板聊,找其他同事抱团开拓新项目,而是守着自己的一亩三分地,企图通过调参提升结果。反观我的mentor,已经在分析完项目成果的上限以后去分析业务的其他增长点了。

3. 没有正视和解决老板的需求

我有一个直接带我的mentor和整个组的老板,有时候会遇到mentor和老板意见不同的时候。一个叫我做东,一个叫我做西。此时我的选择往往是跟着我的mentor,按照他给我制定的详细的方案做一些hard work。而老板私下问我工作进展或提出一些改进方案时,我往往是敷衍了事或是持回避态度。久而久之,老板自然会发现我的不靠谱,也不会信任我或者把新的项目交给我。同时,这样会增加我对mentor的依赖,使得决策能力进一步下降。

大厂经验令人收益

1. 直击痛点的培养模式

总的来说,这段工作经历还是有点tough。除了每周的组会和周报以外,我的mentor还让我写每日工作计划。他甚至会指出我思维上的惯性,一些我从未注意到的陋习:提问稀碎、作息不规律、依赖性强。记得在学校里的时候,我经常在课后找老师问问题,还为自己能提出问题而沾沾自喜。但是现在反思起来,一些稀碎的问题真的值得问吗?能通过自己去查阅资料获得吗?经过自己认真的思考了吗?如果要提出一个有价值的问题,势必要先自我解决掉一些不值一提的问题。

2. 培养目标感和经营思维,参与决策,减少 hard work

对于初级员工而言,就是在不断执行上级的命令。但如果把自己局限于此,并乐于 hard work,实际上是很难获得真正的成长的。迷失于细节,往往要消耗大量的时间。而时间是一种稀缺资源,一旦大量用于细节上的试错,便会减少决策和调研,使得项目不能如期完成。在写文档的时候也需要注意目标感,将目标用一句话写出来,后面的实验围绕着目标铺开,不做漫无目的的数据展示。正如老板的名言,“要学会偷懒。漫无目的不可能带来产出,这些时间为啥不去摸鱼呢?”

3. 基础的数据分析先行

公司的实际环境与数据不同于实验室,可能有大量的重复和无效的干扰数据,需要在前期就处理干净。分析数据的途径也多种多样,需要根据业务确认哪些数据是重点观察的对象。如果数据本身没有含义,再厉害的模型也不能建模出意义。

裁员来势汹汹,我在当天就签署了劳动解除合同,公司承诺我N+1的赔偿(月薪按照当地月平均工资的3倍计算)。对于本地人而言,可以边躺边找下一份工了。未来不知道会如何,也许这会是我人生的一个分叉点。欢迎交流和讨论~

谢谢各位~已经找到下一份工作,是老板内推的,所以异常顺利,涨薪10\%,相当于普调了。老板不忘继续教育……

