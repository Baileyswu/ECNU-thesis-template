\usepackage[colorlinks,linkcolor=black,anchorcolor=blue,citecolor=blue]{hyperref}
\usepackage{geometry}
\usepackage{multirow,multicol}
\usepackage{listings}
\usepackage{xcolor,color}
\usepackage{pdfpages}
\usepackage{shortvrb,ulem,makeidx}
\usepackage{indentfirst,latexsym,amsthm,colortbl,subfigure,clrscode}
\usepackage{algorithm,algorithmic}
\usepackage{amsmath,amssymb,mathrsfs,bm}                % AMSLaTeX宏包 用来排出更加漂亮的公式
\usepackage[subnum]{cases}
\usepackage{enumerate,tabularx}
\usepackage{graphics}
\usepackage{times,fontspec,libertine}
\usepackage{titletoc,bigdelim}
\usepackage{epstopdf,epsfig}
\usepackage{bookmark,booktabs}
\usepackage[bottom]{footmisc}  % footnote 靠底部
\usepackage{perpage}     % footnote 每页重新编号
\usepackage{gbt7714} %默认cite上标
\usepackage[labelsep=space,skip=0.1em]{caption}
\usepackage{enumitem}


%                盲审遮蔽为1,终稿为0    \ano{text}          %
% =========================================================
\def \anonymous {0}

%                equation define                        %
% =========================================================
\DeclareMathOperator*{\argmax}{argmax}
\DeclareMathOperator*{\argmin}{argmin}

\newcommand{\softmax}{\operatorname{softmax}}
\newcommand{\KL}{\operatorname{KL}}
\newcommand{\ano}[1]{\if\anonymous1{xxx}\else{#1}\fi}
\newcommand{\anos}[2]{\if\anonymous1{#2}\else{#1}\fi}

\renewcommand{\algorithmicrequire}{\textbf{Input:}}
\renewcommand{\algorithmicensure}{\textbf{Output:}}
\renewcommand{\algorithmiccomment}[1]{// #1}


%          移除不同章节图表目录之间的空隙                     %
% =========================================================
\newcommand*{\noaddvspace}{\renewcommand*{\addvspace}[1]{}}
\addtocontents{lof}{\protect\noaddvspace}
\addtocontents{lot}{\protect\noaddvspace}
\setlength{\bibsep}{0pt}    %vertical spacing between references
\setenumerate[1]{itemsep=0pt,partopsep=0pt,parsep=\parskip,topsep=0pt}
\setitemize[1]{itemsep=0pt,partopsep=0pt,parsep=\parskip,topsep=0pt}


%%%%% ===== 浮动图表的标题
\DeclareCaptionFormat{mycaption}{\wuhao {\heiti #1}#2~{\heiti #3}}
\captionsetup{format=mycaption}%,belowskip=-10pt


%                color define                            %
% =========================================================
\definecolor{DarkTurquoise}{RGB}{0,206,209}
\definecolor{darkcyan}{RGB}{0,128,128}
\definecolor{gray}{RGB}{130,130,130}
\definecolor{snow}{RGB}{250,250,250}
\definecolor{newton}{RGB}{242,242,242}


%                footnote 每页重新编号                      %
% =========================================================
\MakePerPage{footnote}


%                    根据自己正文需要做的一些定义                 %
%==================================================================
\def\diag{{\rm diag}}
\def\rank{{\rm rank}}
\def\RR{{\cal R}}
\def\NN{{\cal N}}
\def\R{{\mathbb R}}
\def\C{{\mathbb C}}
\let\dis=\displaystyle

\def\p{\partial}
\def\f{\frac}
\def\mr{\mathrm}
\def\mb{\mathbf}
\def\mc{\mathcal}
\def\b{\begin}
\def\e{\end}
\def\vec{\mathbf}

\newtheorem{thm1}{Theorem}[part]
\newtheorem{thm2}{Theorem}[section]
\newtheorem{thm3}{Theorem}[subsection]
\newtheorem{them}[thm2]{定理}
\newtheorem{theorem}[thm2]{定理}
\newtheorem{defn}[thm2]{定义}
\newtheorem{define}[thm2]{定义}
\newtheorem{ex}[thm2]{例}
\newtheorem{exs}[thm2]{例}
\newtheorem{example}[thm2]{例}
\newtheorem{prop}[thm2]{命题}
\newtheorem{lemma}[thm2]{引理}
\newtheorem{cor}[thm2]{推论}
\newtheorem{remark}[thm2]{注释}
\newtheorem{notation}[thm2]{记号}
\newtheorem{abbre}[thm2]{缩写}
% \newtheorem{algorithm}[thm2]{算法}
\newtheorem{problem}[thm2]{问题}
\newtheorem{analysis}{Analysis}
\newcommand{\gameauth}{\mathsf{Game}^\mathsf{Auth}}
\newcommand{\gametul}{\mathsf{Game}^\mathsf{TUL}}
\newcommand{\gamepriv}{\mathsf{Game}^\mathsf{PRIV}}
\newcommand{\advaauth}{\mathsf{Adv}_{\mc A}^{\mathsf{Auth}}}
\newcommand{\advatul}{\mathsf{Adv}_{\mc A}^{\mathsf{TUL}}}
\newcommand{\advapriv}{\mathsf{Adv}_{\mc A}^{\mathsf{PRIV}}}
\newcommand{\tsf}{\textsf}
\newcommand{\be}{\begin{enumerate}}
\newcommand{\ed}{\end{enumerate}}

\newcommand{\tabincell}[2]{\begin{tabular}{@{}#1@{}}#2\end{tabular}}
\newcommand{\mbb}{\mathbb}

\newcommand{\yihao}{\fontsize{26pt}{36pt}\selectfont}           % 一号, 1.4 倍行距
\newcommand{\erhao}{\fontsize{22pt}{28pt}\selectfont}          % 二号, 1.25倍行距
\newcommand{\xiaoer}{\fontsize{18pt}{18pt}\selectfont}          % 小二, 单倍行距
\newcommand{\sanhao}{\fontsize{16pt}{24pt}\selectfont}        % 三号, 1.5倍行距
\newcommand{\xiaosan}{\fontsize{15pt}{22pt}\selectfont}        % 小三, 1.5倍行距
\newcommand{\sihao}{\fontsize{14pt}{21pt}\selectfont}            % 四号, 1.5 倍行距
\newcommand{\banxiaosi}{\fontsize{13pt}{19.5pt}\selectfont}    % 半小四, 1.5倍行距
\newcommand{\xiaosi}{\fontsize{12pt}{18pt}\selectfont}            % 小四, 1.5倍行距
\newcommand{\dawuhao}{\fontsize{11pt}{11pt}\selectfont}       % 大五号, 单倍行距
\newcommand{\wuhao}{\fontsize{10.5pt}{15.75pt}\selectfont}    % 五号, 单倍行距

\ctexset{
    chapter = {
        format+={\zihao{3}\heiti},
        afterskip={12pt},
        beforeskip={-12pt},
    },
    section = {
        format+={\raggedright\zihao{4}\heiti},
    },
    subsection = {
        format+={\zihao{-4}\heiti},
    },
}


\lstset
{
	basicstyle=\ttfamily,
	% numbers=left,
	% numberstyle=\tiny,
	keywordstyle=\color[RGB]{0, 0, 255},
	commentstyle=\color[RGB]{0, 128, 0},
	frame=shadowbox,
	rulesepcolor=\color{red!20!green!20!blue!20},
	showspaces=false,
	showstringspaces=false,
	extendedchars=false,
	showtabs=false,
	tabsize=4,
	xleftmargin=0.5em,
	xrightmargin=0.5em,
	% aboveskip=1em,
	escapeinside=``
}

%============================= 页面设置 ================================%
%-------------------- 定义页眉和页脚 使用fancyhdr 宏包 -----------------%
% 定义页眉与正文间双隔线
% http://ctan.math.washington.edu/tex-archive/macros/latex/contrib/fancyhdr/fancyhdr.pdf

\fancypagestyle{mainFancy}{
    \fancyhf{}
    \fancyhead[C]{\footnotesize{华东师范大学硕士学位论文}}
	\fancyfoot[C]{\footnotesize\thepage}

    \newcommand{\makeheadrule}{%
		\rule[0.85\baselineskip]{\headwidth}{0.9pt}\vskip-1.1\baselineskip %headwidth 为线的粗细 vskip为线以baseline为基础的偏移,上+下-
    }

    \makeatletter
    \renewcommand{\headrule}{%
        {\if@fancyplain\let\headrulewidth\plainheadrulewidth\fi
        \makeheadrule}
    } 
	\makeatother
	
	\newcommand{\adots}{\mathinner{\mkern 2mu%
	\raisebox{0.1em}{.}\mkern 2mu\raisebox{0.4em}{.}%
	\mkern2mu\raisebox{0.7em}{.}\mkern 1mu}}

	\setmainfont{Times New Roman}
    % \dottedcontents{⟨section ⟩}[⟨left ⟩]{⟨above-code ⟩} {⟨label width⟩}{⟨leader width⟩}
	\dottedcontents{chapter}[1.8cm]{\sihao\heiti}{3.8em}{6pt}  %目录中的字体
    \dottedcontents{section}[1.8cm]{\xiaosi\heiti}{2.8em}{6pt}
    \dottedcontents{subsection}[3.0cm]{\xiaosi}{2.8em}{6pt}
}

% 每章首页
\fancypagestyle{plain}{
    \fancyhf{}
    \fancyhead[C]{\footnotesize{华东师范大学硕士学位论文}}
	\fancyfoot[C]{\footnotesize\thepage}

    \makeatletter
    \renewcommand{\headrule}{%
        {\if@fancyplain\let\headrulewidth\plainheadrulewidth\fi
        \makeheadrule}
    } 
	\makeatother
}